%-------------------------
% Resume in LaTeX
% Author : Murat Raimbekov
% Based on: Jake's Resume (https://github.com/jakeryang/resume)
% License : MIT
%------------------------

\documentclass[letterpaper,11pt]{article}

\usepackage{latexsym}
\usepackage[empty]{fullpage}
\usepackage{titlesec}
\usepackage{marvosym}
\usepackage[usenames,dvipsnames]{color}
\usepackage{verbatim}
\usepackage{enumitem}
\usepackage[hidelinks]{hyperref}
\usepackage{fancyhdr}
\usepackage[english]{babel}
\usepackage{tabularx}
\usepackage{fontawesome5}
\usepackage{multicol}
\setlength{\multicolsep}{-3.0pt}
\setlength{\columnsep}{-1pt}
\input{glyphtounicode}

\pagestyle{fancy}
\fancyhf{}
\fancyfoot{}
\renewcommand{\headrulewidth}{0pt}
\renewcommand{\footrulewidth}{0pt}

% Adjust margins
\addtolength{\oddsidemargin}{-0.6in}
\addtolength{\evensidemargin}{-0.5in}
\addtolength{\textwidth}{1.19in}
\addtolength{\topmargin}{-.7in}
\addtolength{\textheight}{1.4in}

\urlstyle{same}

\raggedbottom
\raggedright
\setlength{\tabcolsep}{0in}

% Sections formatting
\titleformat{\section}{
  \vspace{-4pt}\scshape\raggedright\large\bfseries
}{}{0em}{}[\color{black}\titlerule \vspace{-5pt}]

% Ensure that generate pdf is machine readable/ATS parsable
\pdfgentounicode=1

%-------------------------
% Custom commands
\newcommand{\resumeItem}[1]{
  \item\small{
    {#1 \vspace{-2pt}}
  }
}

\newcommand{\classesList}[4]{
    \item\small{
        {#1 #2 #3 #4 \vspace{-2pt}}
  }
}

\newcommand{\resumeSubheading}[4]{
  \vspace{-2pt}\item
    \begin{tabular*}{1.0\textwidth}[t]{l@{\extracolsep{\fill}}r}
      \textbf{#1} & \textbf{\small #2} \\
      \textit{\small#3} & \textit{\small #4} \\
    \end{tabular*}\vspace{-7pt}
}

\newcommand{\resumeSubSubheading}[2]{
    \item
    \begin{tabular*}{0.97\textwidth}{l@{\extracolsep{\fill}}r}
      \textit{\small#1} & \textit{\small #2} \\
    \end{tabular*}\vspace{-7pt}
}

\newcommand{\resumeProjectHeading}[2]{
    \item
    \begin{tabular*}{1.001\textwidth}{l@{\extracolsep{\fill}}r}
      \small#1 & \textbf{\small #2}\\
    \end{tabular*}\vspace{-7pt}
}

\newcommand{\resumeSubItem}[1]{\resumeItem{#1}\vspace{-4pt}}

\renewcommand\labelitemi{$\vcenter{\hbox{\tiny$\bullet$}}$}
\renewcommand\labelitemii{$\vcenter{\hbox{\tiny$\bullet$}}$}

\newcommand{\resumeSubHeadingListStart}{\begin{itemize}[leftmargin=0.0in, label={}]}
\newcommand{\resumeSubHeadingListEnd}{\end{itemize}}
\newcommand{\resumeItemListStart}{\begin{itemize}}
\newcommand{\resumeItemListEnd}{\end{itemize}\vspace{-5pt}}

%-------------------------------------------
%%%%%%  RESUME STARTS HERE  %%%%%%%%%%%%%%%%%%%%%%%%%%%%

\begin{document}

%----------HEADING----------
\begin{center}
    {\Huge \scshape Murat Raimbekov} \\ \vspace{1pt}
    Bishkek, Kyrgyzstan \\ \vspace{1pt}
    \small \raisebox{-0.1\height}\faPhone\ +996 (772) 109-103 ~
    \href{mailto:raimbekov_m@auca.kg}{\raisebox{-0.2\height}\faEnvelope\ \underline{raimbekov\_m@auca.kg}} ~
    \href{https://www.linkedin.com/in/murat-raimbekov}{\raisebox{-0.2\height}\faLinkedin\ \underline{Murat Raimbekov}} ~
    \href{https://github.com/raimbekovm}{\raisebox{-0.2\height}\faGithub\ \underline{raimbekovm}}
    \vspace{-8pt}
\end{center}

%-----------EDUCATION-----------
\section{Education}
  \resumeSubHeadingListStart
    \resumeSubheading
      {American University of Central Asia}{Bishkek, Kyrgyzstan}
      {Bachelor's degree in Applied Mathematics and Informatics}{Sep 2022 -- May 2026}
      \resumeItemListStart
        \resumeItem{Relevant Coursework: Neural Networks and Deep Learning (CV), Data Science and Machine Learning, OOP, Data Structures \& Algorithms, The Theory of Probabilities and Statistics, Mathematical Analysis I \& II}
        \resumeItem{GPA: \textbf{3.3/4}}
      \resumeItemListEnd
  \resumeSubHeadingListEnd

%-----------TECHNICAL SKILLS-----------
\section{Technical Skills}
 \begin{itemize}[leftmargin=0.15in, label={}]
    \small{\item{
     \textbf{Languages}{: Python, C++, SQL (PostgreSQL)} \\
     \textbf{Computer Vision}{: PyTorch, TorchVision, OpenCV, Albumentations, scikit-image, Object Detection (YOLO), Image Segmentation (U-Net, SegNet)} \\
     \textbf{Machine Learning \& Data}{: scikit-learn, XGBoost, LightGBM, pandas, NumPy, Matplotlib, Seaborn} \\
     \textbf{Developer Tools}{: Git, GitHub, GitLab, Jupyter Notebook}
    }}
 \end{itemize}
 \vspace{-16pt}

%-----------EXPERIENCE-----------
\section{Experience}
  \resumeSubHeadingListStart

    \resumeSubheading
      {Data Science Intern}{Jul 2025 -- Present}
      {Baker Tilly}{Bishkek, Kyrgyzstan}
      \resumeItemListStart
        \resumeItem{Contributed to the development of \textbf{\href{https://www.linkedin.com/company/atlas-kyrgyzstan/}{Atlas}}, a digital startup platform for automated assessment of real estate and movable assets in Kyrgyzstan, helping scale automated asset valuation for \textbf{500+ properties}}
        \resumeItem{Performed data collection, cleaning, and feature engineering on \textbf{10,000+ property records}, reducing data errors by \textbf{7\%}}
        \resumeItem{Fine-tuned \textbf{Gradient Boosting Decision Trees (XGBoost, LightGBM)} achieving high valuation accuracy}
        \resumeItem{Collaborated with 3 cross-functional teams (legal, finance, operations) to align ML models with business requirements}
      \resumeItemListEnd

  \resumeSubHeadingListEnd
\vspace{-16pt}

%-----------PROJECTS-----------
\section{Projects}
    \vspace{-5pt}
    \resumeSubHeadingListStart

      \resumeProjectHeading
          {\textbf{Computer Vision Road Defects Detection} $|$ \emph{Yandex Hackathon}}{Dec 2025}
          \resumeItemListStart
            \resumeItem{Developed an automated road defect detection system using \textbf{YOLOv8} for real-time object detection on highway images}
            \resumeItem{Implemented data augmentation pipeline with \textbf{Albumentations} to improve model robustness on diverse road conditions}
            \resumeItem{Achieved \textbf{85\%+ mAP} on test set by fine-tuning pre-trained models and applying transfer learning techniques}
            \resumeItem{Utilized \textbf{OpenCV} for image preprocessing and post-processing to enhance detection accuracy}
            \resumeItem{Deployed model inference pipeline with \textbf{PyTorch} for batch processing of highway surveillance footage}
          \resumeItemListEnd
          \vspace{-13pt}

      \resumeProjectHeading
          {\textbf{Medical Image Segmentation} $|$ \emph{PyTorch, U-Net, SegNet} $|$ \href{https://github.com/raimbekovm/computer_vision_course}{\underline{GitHub}}}{Sep 2025}
          \resumeItemListStart
            \resumeItem{Implemented semantic segmentation models (\textbf{U-Net, SegNet, Residual U-Net}) for skin lesion detection on PH2 dataset}
            \resumeItem{Achieved \textbf{0.654 IoU} with SegNet architecture, optimizing BCE, Dice, and Focal loss functions}
            \resumeItem{Conducted comprehensive ablation studies comparing architectures and loss functions with visualization analysis}
            \resumeItem{Applied data augmentation techniques using \textbf{scikit-image} and custom preprocessing pipelines}
          \resumeItemListEnd
          \vspace{-13pt}

      \resumeProjectHeading
          {\textbf{Grain Classification} $|$ \emph{PyTorch, EfficientNet, ConvNeXt} $|$ \href{https://github.com/raimbekovm/computer_vision_course}{\underline{GitHub}}}{Oct 2025}
          \resumeItemListStart
            \resumeItem{Built multi-class image classifier using \textbf{timm} library with EfficientNetV2 and ConvNeXt architectures}
            \resumeItem{Implemented 5-fold cross-validation and ensemble methods with Test-Time Augmentation (TTA) for robust predictions}
            \resumeItem{Optimized training pipeline with mixed precision (\textbf{torch.cuda.amp}) and learning rate scheduling}
            \resumeItem{Processed and augmented 1000+ images using custom \textbf{PyTorch Dataset} and \textbf{DataLoader} implementations}
          \resumeItemListEnd
          \vspace{-13pt}

      \resumeProjectHeading
          {\textbf{Insulator Segmentation} $|$ \emph{PyTorch, U-Net, ResNet34} $|$ \href{https://github.com/raimbekovm/computer_vision_course}{\underline{GitHub}}}{Nov 2025}
          \resumeItemListStart
            \resumeItem{Developed semantic segmentation model for power line insulator detection achieving \textbf{0.9895 Dice coefficient}}
            \resumeItem{Leveraged transfer learning with \textbf{ResNet34 encoder} and implemented U-Net decoder architecture}
            \resumeItem{Applied data augmentation strategies and Test-Time Augmentation using \textbf{Albumentations} library}
          \resumeItemListEnd

    \resumeSubHeadingListEnd

\end{document}
